\documentclass[a4paper,12pt]{article}
\usepackage{etex}
\usepackage{fancyhdr}
\usepackage{fancyheadings}
\usepackage[ngerman,german]{babel}
\usepackage{german}
\usepackage[utf8]{inputenc}
%\usepackage[latin1]{inputenc}
\usepackage[active]{srcltx}
\usepackage{algorithm}
\usepackage[noend]{algorithmic}
\usepackage{amsmath}
\usepackage{amssymb}
\usepackage{pifont}
\usepackage{amsthm}
\usepackage{bbm}
\usepackage{enumerate}
\usepackage{stmaryrd}
\usepackage{tabu}
\usepackage{graphicx}
\usepackage{subcaption}
\usepackage{ifthen}
\usepackage{listings}
\usepackage{struktex}
\usepackage{booktabs}
\usepackage{hyperref}
\usepackage{upgreek}
\usepackage{tikz}
\usepackage{marvosym}
\usepackage{bussproofs}
\usepackage{multirow}
\usepackage{bigdelim}
\usepackage{tikz-qtree}
\usetikzlibrary{positioning}

%%%%%%%%%%%%%%%%%%%%%%%%%%%%%%%%%%%%%%%%%%%%%%%%%%%%%%
%%%%%%%%%%%%%% EDIT THIS PART %%%%%%%%%%%%%%%%%%%%%%%%
%%%%%%%%%%%%%%%%%%%%%%%%%%%%%%%%%%%%%%%%%%%%%%%%%%%%%%
\newcommand{\Fach}{Sortieralgorithmen - Heapsort}
\newcommand{\Name}{Hien Nguyen}
%\newcommand{\Matrikelnummer}{3746594 \& ???}
%\newcommand{\Fachsemester}{6}
\newcommand{\Semester}{Abgabe: 24.06.2015}
\newcommand{\Uebungsblatt}{} %  <-- UPDATE ME
%%%%%%%%%%%%%%%%%%%%%%%%%%%%%%%%%%%%%%%%%%%%%%%%%%%%%%
%%%%%%%%%%%%%%%%%%%%%%%%%%%%%%%%%%%%%%%%%%%%%%%%%%%%%%

\setlength{\parindent}{0em}
\topmargin -1.0cm
\oddsidemargin 0cm
\evensidemargin 0cm
\setlength{\textheight}{9.2in}
\setlength{\textwidth}{6.0in}

%%%%%%%%%%%%%%%
%% Aufgaben-COMMAND
\newcommand{\Aufgabe}[1]{
	{
		\vspace*{0.5cm}
		\textsf{\textbf{Aufgabe #1}}
		\vspace*{0.2cm}
		
	}
}
%%%%%%%%%%%%%%
\hypersetup{
	pdftitle={\Fach{}: "Ubungsblatt \Uebungsblatt{}},
	pdfauthor={\Name},
	pdfborder={0 0 0}
}

\lstset{ %
	language=java,
	basicstyle=\footnotesize\tt,
	showtabs=false,
	tabsize=2,
	captionpos=b,
	breaklines=true,
	extendedchars=true,
	showstringspaces=false,
	flexiblecolumns=true,
}

\title{"Ubungsblatt \Uebungsblatt{}}
\author{\Name{}}

\begin{document}
	\thispagestyle{fancy}
	\lhead{\sf \large \Fach{} \\ \small \Name{} }%- \Matrikelnummer{}}
	\rhead{\sf \Semester{} \\ }%Fachsemester \Fachsemester{}}
	\vspace*{0.2cm}
	\begin{center}
		\LARGE \sf \textbf{"Ubungsblatt \Uebungsblatt{}}
	\end{center}
	\vspace*{0.2cm}
	
	% % % % % % % % %
	\renewcommand{\labelenumi}{\alph{enumi})}
	\renewcommand{\labelenumii}{\arabic{enumii})}
	\newcommand{\aeq}{%	
		\mathrel{\reflectbox{\rotatebox[origin=c]{0}{$\models$}}}} 
	
	% % % % % % % %
	\Aufgabe{1. generateHeap}
	Gegeben ist im folgenden eine Zahlenfolge, erzeugen Sie einen Min-Heap und geben ihre Schritte stichwortartig an.
\begin{table}[h]
\centering
\begin{tabular}{lllllll}
7 & 3 & 5 & 5 & 6 & 0 & 8
\end{tabular}
\end{table}
\\\\\\\\\\\\\\\\\\\\\\\\\\\\\\\\\\
\Aufgabe{2. sortHeap}
Sortieren Sie schrittweise, den aus Aufgabe 1 erzeugten Min-Heap.
Falls Sie sich nicht sicher sind, nehmen sie diesen Min-Heap:
\begin{table}[h]
\centering
\begin{tabular}{lllllll}
0 & 3 & 5 & 5 & 6 & 7 & 8
\end{tabular}
\end{table}\\\\\\\\\\\

\Aufgabe{(Zusatz) Min-Heap Implementation}
Laden Sie sich von der Moodle Plattform die Quelldateien runter und Implementieren Sie einen MinHeap.
\end{document}